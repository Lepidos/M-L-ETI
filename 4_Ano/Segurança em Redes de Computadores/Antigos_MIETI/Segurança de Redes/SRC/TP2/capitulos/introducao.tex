\documentclass[../tp2.tex]{subfiles}

\begin{document}
O mecanismo de controlo de acesso visa garantir o acesso autorizado a determinada propriedade. Este mecanismo é composto por três processos distintos, a autenticação, autorização e auditoria.\par 
A autenticação trata-se da identificação do utilizador que pretende ter acesso ao sistema através de credenciais, como por exemplo um endereço eletrónico e a respetiva palavra-chave. A autorização define os direitos e permissões que o utilizador tem no sistema que implementa o controlo de acesso. Por fim a auditoria permite através da análise de dados relativos à utilização dos recurso do sistema, concluir a natureza dessa mesma utilização.\par 
Estes processos são utilizados consoante as técnicas de controlo acesso implementadas, estas técnicas podem ser discricionárias, obrigatórias ou \textit{Role-Based}. O controlo de acesso discricionário é uma política de controlo de acesso determinada pelo proprietário do recurso. É este que atribui as permissões de acesso à sua propriedade. Por outro lado a política de controlo de acesso obrigatório é o sistema que determina as propriedades de acesso aos seus recursos e não os proprietários desses mesmos recursos. Existe uma terceira política, o controlo baseado em papéis, esta define os direitos e permissões consoante o papel que determinado utilizador tem dentro da sua organização. Esta técnica visa a simplificar a gestão de permissões dadas aos utilizadores dentro da mesma organização.\par 
Para este trabalho prático é requerido a conceptualização de um modelo de controlo de acesso no contexto académico tendo como referência o modelo Bell - LaPadula.
\end{document}