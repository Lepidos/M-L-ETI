\documentclass[../momento_1.tex]{subfiles}

\begin{document}

O foco principal deste projeto prático é o desenvolvimento de um agente SNMP que seja um servidor que gera números aleatórios. Este tipo de serviço remoto através da \textit{Internet} está disponível a diferentes sistemas que necessitem de formas de obter números aleatórios não controlados por processos internos ao próprio sistema computacional, como por exemplo o serviço \textit{random.org} cuja interface comunicacional é efetuada através de HTTP. O objetivo deste trabalho consiste na implementação de um serviço que possua uma interface comunicacional através de SNMPv2c e não através de HTTP.\par
O primeiro passo realizado foi a definição dos requisitos funcionais, ou seja, o tipo de
resultados esperados tendo em conta as possíveis parametrizações que os utilizadores podem realizar.\par 
Após o primeiro passo estar completo, o objetivo seguinte passou pela definição de uma MIB com os grupos de objetos com a semântica e sintaxe adequadas à correta abstração dos requisitos funcionais estabelecidos anteriormente. \par
Por fim, o foco de trabalho consistiu na construção e teste do software do agente que implementa o serviço num agente SNMPv2c, desenvolvido na linguagem de programação Java.\\[5cm]



\end{document}
