\documentclass[../momento_1.tex]{subfiles}


\begin{document}

\par No âmbito da unidade curricular de Gestão de Redes, do curso de Engenharia de Telecomunicações e Informática, foi proposto aos alunos a realização de três trabalhos práticos. Este relatório diz respeito ao último trabalho prático, o número 3.\par
O principal objetivo deste trabalho prático, consiste em consolidar os conhecimentos adquiridos na unidade curricular, nomeadamente sobre a utilização prática do modelo de gestão
preconizado pelo INMF (\textit{Internet Standard Management Framework}) que engloba
componentes como o protocolo SNMP (\textit{Simple Network Management Protocol}) e as
MIBs (\textit{Management Information Bases}).\par
Foi pedido aos alunos a conceção de agente SNMP que seja um servidor de geração de números aleatórios cuja interface comunicacional seja através de SNMPv2c numa linguagem à escolha, no nosso caso foi usado a linguagem Java.\par
Para a resolução deste projeto foi necessário a utilização de APIs SNMP e diversas ferramentas adicionais que nos permitem a construção do que é requerido no enunciado, ou seja, a implementação de uma ferramenta de gestão.\\[2cm]



 
\setlength{\epigraphwidth}{4in} 
\epigraph{\textit{"if you need motivation don't do it"}}{Elon Musk}

\end{document}

