\documentclass[../momento_1.tex]{subfiles}

\begin{document}

O pacote de software SNMP a instalar disponibiliza um agente SNMP para Unix. Pretende-se
nesta secção a configuração e ativação de um agente SNMP na estação de trabalho. Para isso, é necessário estudar as páginas do manual (\textit{manpages}) do \textit{daemon snmpd} e do ficheiro de configuração \textit{snmpd.conf} e proceder à ativação do agente na porta 5555.
Experimente o software instalado, obtendo, nomeadamente, a seguinte informação de monitorização
(valores de variáveis da MIB-2):

\begin{itemize}
\item  Os valores das instâncias de todas as variáveis do grupo system da MIB-2 da sua estação de
trabalho e de um qualquer encaminhador IP (um qualquer \textit{router} da rede da Universidade do
Minho ou um da sua rede doméstica);
\item Informações dos interfaces de rede da sua estação de trabalho e de um qualquer encaminhador
IP (um qualquer \textit{router} da rede da Universidade do Minho ou um da sua rede doméstica).
\end{itemize}

\end{document}