\documentclass[../momento_1.tex]{subfiles}

\begin{document}
\par De forma a resolver esta questão, foi pedido aos alunos para consultarem a bibliografia disponível sobre o Cisco IOS e descobrir os comandos necessários para efetivar os seguintes requisitos:

\begin{itemize}
\item Permissão de leitura pública dos valores das instâncias de todos os objetos da MIB-2; 
\item Permissão de alteração das instâncias dos objetos do grupo snmp para a comunidade
gestaoderedes20162017, a partir da rede local.
\end{itemize}

\par Após uma leitura cuidada da bibliografia referida acima, cheguei à conclusão que os comandos
necessários para a implementação do que é requerido são os seguintes, representados no Exemplo \ref{lst:comandos}.\par

{\setstretch{1.1}
\begin{lstlisting}[caption={Comandos para implementação das permissões.},label={lst:comandos},language=JAVA]
	snmp-server view
	snmp-server community
\end{lstlisting}}

\setstretch{1.1}
\par Através do estudo percebi que era necessário recorrer a “view’s” para permitir a uma determinada comunidade obter acesso a um ramo de OIDs de uma árvore SNMP. Este comando permite criar ou atualizar uma “view”. Também é possível remover uma "view" introduzindo "no" no inicio do comando. A sintaxe pode ser visualizada no Exemplo \ref{lst:sintax}. \par

{\setstretch{1.1}
\begin{lstlisting}[caption={Sintaxe do comando snmp-server view.},label={lst:sintax},language=JAVA]
	snmp-server view view-name oid-tree included | excluded
	no snmp-server view view-name
\end{lstlisting}}

Na Tabela \ref{tabela} podemos ver em detalhe o que significa cada uma das
partes constituintes do comando.\par


\begin{table}[H]
\centering
\caption{Significado do comando snmp-server view}
\label{tabela}
\begin{tabular}{|l|l|}
\hline
view-name           & Identifica o nome da view que está a ser criada ou atualizada.     \\ \hline
oid-tree            & Especifica o OID da subtree para ser incluída ou excluída da view. \\ \hline
included | excluded & Indica se o tipo de view é incluído ou excluído.                   \\ \hline
\end{tabular}
\end{table}

\par Quanto ao segundo comando referido acima no Exemplo \ref{lst:comandos}, é utilizado para configurar a \textit{community string} de acesso ao protocolo SNMP. Tal como o comando anterior, podemos eliminar uma \textit{community string} sendo para isso necessário introduzir “no”
no inicio do comando. A respetiva sintaxe pode ser visualizada no Exemplo \ref{lst:lel}.\par 

{\setstretch{1.1}
\begin{lstlisting}[caption={Sintaxe do comando snmp-server community.},label={lst:lel},language=JAVA]
	snmp-server community string [view view-name] [ro | rw] [number]
\end{lstlisting}}

\par A Tabela \ref{tabelaW} apresenta as opções que podem ser usadas.

\begin{table}[H]
\centering
\caption{Significado do comando snmp-server community}
\label{tabelaW}
\begin{tabular}{|l|l|}
\hline
string    & Atua como password e permite o acesso ao protocolo SNMP.                                                                                                                                                \\ \hline
view-name & Nome da view definida anteriormente. (opcional)                                                                                                                                                         \\ \hline
ro | rw   & \begin{tabular}[c]{@{}l@{}}Especifica o tipo de acesso para leitura ou para leitura-escrita, \\ caso seja utilizado a opção de rw. (opcional)\end{tabular}                                              \\ \hline
number    & \begin{tabular}[c]{@{}l@{}}Inteiro de 1 a 99 que especifica uma tabela de IP's que que estão autorizados \\ a utilizar a string de comunidade para obter acesso ao agente SNMP. (opcional)\end{tabular} \\ \hline
\end{tabular}
\end{table}

\par Concluindo, de forma a responder corretamente ao enunciado desta questão, os comandos finais estão apresentados no Exemplo \ref{lst:final}.

{\setstretch{1.1}
\begin{lstlisting}[caption={Comandos para responder ao enunciado.},label={lst:final},language=JAVA]
	snmp-server view view1 mib-2 included
	snmp-server community public view view1 ro
    	snmp-server view view2 snmp included
    	snmp-server community gestaoderedes20162017 view view2 rw
\end{lstlisting}}

\par O primeiro comando cria uma "view" de nome view1 com acesso ao grupo mib-2. Já o segundo, cria a comunidade pública que tem acesso apenas de leitura à view1. O terceiro comando cria uma "view" de nome view2 com o grupo snmp. Por fim, cria-se uma comunidade de nome gestaoderedes20162017 com acesso de leitura e escrita à view2.
\end{document}